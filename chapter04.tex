\chapter{Деректер құрылымдары}

\index{деректер құрылымы}

\key{Деректер құрылымы} -- деректі компьютер 
жадында сақтау жолы.
Деректер құрылымын есепке сай қолдана білудің маңызы зор.
Себебі олардың әрқайсысының өз артықшылықтары мен кемшіліктері болады.
Осыдан келіп, ''таңдалған деректер құрылымы үшін қандай
операциялар тиімді?'' - деген сұрақ туындайды.
%The crucial question is: which operations
%are efficient in the chosen data structure?

Бұл бөлім C++ стандартты дерекханасының  
ең маңызды деректер құрылымдарымен таныстырады.
Көп уақыт үнемдейтіндіктен стандартты дерекхананы 
қолдану қашанда жақсы идея саналады.
Біз стандартты дерекханада жоқ күрделірек
деректер құрылымдарын кейінірек қарастырамыз.

\section{Динамикалық жиымдар}

\index{динамикалық жиым}
\index{вектор}
\key{Динамикалық жиым} -- бағдарлама барысында
өлшемі өзгере алатын жиым.
С++-те кең қолданыстағы динамикалық жиым -- кәдімгі
жиым ретінде де пайдалануға болатын вектор құрылымы.

Төмендегі кодта бос вектор құрылып, оған 3 элемент 
қосылады:

\begin{lstlisting}
vector<int> v;
v.push_back(3); // [3]
v.push_back(2); // [3,2]
v.push_back(5); // [3,2,5]
\end{lstlisting}

Содан соң элементтерді кәдімгі жиымдағыдай ала аламыз:

\begin{lstlisting}
cout << v[0] << "\n"; // 3
cout << v[1] << "\n"; // 2
cout << v[2] << "\n"; // 5
\end{lstlisting}

\texttt{size} функциясы вектордағы элементтер санын қайтарады.
Төмендегі код вектор бойынша өтіп шығып, барлық элементтерін қамтиды:

\begin{lstlisting}
for (int i = 0; i < v.size(); i++) {
    cout << v[i] << "\n";
}
\end{lstlisting}

\begin{samepage}
Векторды өтіп шығудың қысқаша жолы:

\begin{lstlisting}
for (auto x : v) {
    cout << x << "\n";
}
\end{lstlisting}
\end{samepage}

\texttt{back} функциясы вектордың соңғы элементін қайтарса,
\texttt{pop\_back} функциясы соңғы элементті өшіреді:

\begin{lstlisting}
vector<int> v;
v.push_back(5);
v.push_back(2);
cout << v.back() << "\n"; // 2
v.pop_back();
cout << v.back() << "\n"; // 5
\end{lstlisting}

Келесі код арқылы бес элементтен тұратын вектор құрылады:

\begin{lstlisting}
vector<int> v = {2,4,2,5,1};
\end{lstlisting}

Векторды құрудың тағы бір жолы -- элементтер саны мен олардың
әрқайсысы үшін бастапқы мәнді анықтау. Мысалы:

\begin{lstlisting}
// size 10, initial value 0
vector<int> v(10);
\end{lstlisting}
\begin{lstlisting}
// size 10, initial value 5
vector<int> v(10, 5);
\end{lstlisting}

Вектор орындалу барысында қарапайым жиымды қолданады.
Егер вектордың өлшемі өсіп, жиым тым кішкентай болып қалса,
жаңа жиым ашылып, барлық элементтер соған көшіріледі.
Бұл процесс жиі болмайтындықтан, \texttt{push\_back} орташа уақытша күрделілігі
$O(1)$-ге тең болады.
% The internal implementation of a vector
% uses an ordinary array.
% If the size of the vector increases and
% the array becomes too small,
% a new array is allocated and all the
% elements are moved to the new array.
% However, this does not happen often and the
% average time complexity of
% \texttt{push\_back} is $O(1)$.

\index{string}

\texttt{Жол} (string) құрылымы да вектор сияқты қолданыла алатын
динамикалық жиымға жатады. Сонымен қатар бұл жерде басқа деректер 
құрылымдарында жоқ арнайы синтаксис бар.
Жолдар \texttt{+} символы арқылы біріктіріле алады.
$\texttt{substr}(k,x)$ функциясы $k$ позициясынан басталып,
ұзындығы $x$ болатын ішжолды қайтарады, ал $\texttt{find}(\texttt{t})$
функциясы алғаш кездескен \texttt{t} ішжолының позициясын қайтарады.

Келесі кодта кей жол операциялары көрсетілген:

\begin{lstlisting}
string a = "hatti";
string b = a+a;
cout << b << "\n"; // hattihatti
b[5] = 'v';
cout << b << "\n"; // hattivatti
string c = b.substr(3,4);
cout << c << "\n"; // tiva
\end{lstlisting}

\section{Жиын құрылымдары}

\index{set}

\key{Жиын} (set) -- элементтерді жинақтайтын деректер 
құрылымы. Ал жиындарда орындалатын негізгі операциялар --
енгізу, іздеу мен өшіру.

C++ стандартты дерекханасында жиынның екі
орындалу жолы бар. Олар:
\texttt{set} -- теңгерімделген бинарлы дараққа негізделген құрылым. Оның
операциялары $O(\log n)$ уақытта жұмыс жасайды. Ал екіншісі --
% The C++ standard library contains two set
% implementations:
% The structure \texttt{set} is based on a balanced
% binary tree and its operations work in $O(\log n)$ time.
\texttt{unordered\_set}, мұнда хэш қолданылады және оның операциялары
орта есеппен $O(1)$ уақытта орындалады.

Қай орындалу жолын таңдау көбіне талғамға байланысты болып келеді.
% The choice of which set implementation to use
% is often a matter of taste.
\texttt{set} құрылымының артықшылығы -- элементтерді реттілігі бойынша сақтап,
\texttt{unordered\_set}-те жоқ функцияларды қамтуында.
% The benefit of the \texttt{set} structure
% is that it maintains the order of the elements
% and provides functions that are not available
% in \texttt{unordered\_set}.
Дегенмен, \texttt{unordered\_set} те кей жағдайларда әлдеқайда тиімді болуы мүмкін.

Келесі код бүтін сандарды сақтайтын жиын құрады
және кейбір операцияларды орындайды.
\texttt{insert} функциясы жиынға элементті қосады,
\texttt{count} функциясы элементтің жиында қанша рет кезескенін анықтайды,
ал \texttt{erase} функциясы элементті жиыннан өшіреді.

\begin{lstlisting}
set<int> s;
s.insert(3);
s.insert(2);
s.insert(5);
cout << s.count(3) << "\n"; // 1
cout << s.count(4) << "\n"; // 0
s.erase(3);
s.insert(4);
cout << s.count(3) << "\n"; // 0
cout << s.count(4) << "\n"; // 1
\end{lstlisting}

Жиын көбінесе вектор сияқты қолданыла алады,
бірақ, элементтерге \texttt{[]} арқылы қол жеткізу мүмкін емес.
% A set can be used mostly like a vector,
% but it is not possible to access
% the elements using the \texttt{[]} notation.
Төмендегі код жиын құрайды, оның ішіндегі элементтер санын шығарады,
кейін барлық элементтерді өтіп шығады:
\begin{lstlisting}
set<int> s = {2,5,6,8};
cout << s.size() << "\n"; // 4
for (auto x : s) {
    cout << x << "\n";
}
\end{lstlisting}

Ішіндегі элементтердің барлығының
\emph{өзгеше} болуы жиынның маңызды қасиетіне жатады.
% An important property of sets is
% that all their elements are \emph{distinct}.
Сол себепті \texttt{count} функциясы әрдайым 0 (элемент жиында жоқ)
немесе 1 (элемент жиымда бар) мәндерін қайтарады
және \texttt{insert} элемент жиында бар болған жағдайда
оны ешқашан қайта қоспайды.
Бұл келесі кодта көрсетілген:

\begin{lstlisting}
set<int> s;
s.insert(5);
s.insert(5);
s.insert(5);
cout << s.count(5) << "\n"; // 1
\end{lstlisting}

Сонымен қатар С++-те \texttt{multiset} және 
\texttt{unordered\_multiset} құрылымдары бар. 
Олар \texttt{set} пен \texttt{unordered\_set}
сияқты жұмыс істейді, бірақ элементтің бірнеше 
данасын қамти алады.
Мысалдағы кодта 5 саны multiset-ке қосылғандай үш рет кездеседі:
% For example, in the following code all three instances
% of the number 5 are added to a multiset:

\begin{lstlisting}
multiset<int> s;
s.insert(5);
s.insert(5);
s.insert(5);
cout << s.count(5) << "\n"; // 3
\end{lstlisting}
\texttt{erase} функциясы элементтің 
барлық кездескен сәттерін multiset-тен өшіреді:
\begin{lstlisting}
s.erase(5);
cout << s.count(5) << "\n"; // 0
\end{lstlisting}
Тек бір данасын ғана өшіру қажет болатын жағдай жиі кездеседі.
Оны келесі түрде орындай аламыз:
\begin{lstlisting}
s.erase(s.find(5));
cout << s.count(5) << "\n"; // 2
\end{lstlisting}

\section{Сөздік құрылымы}

\index{сөздік}

\key{Сөздік} (map) -- кілт-мәнді жұптардан тұратын жиым.
% \key{map} is a generalized array
% that consists of key-value-pairs.
Қарапайым $n$ өлшемді жиым үшін кілттер 
әрдайым жүйелі $0,1,\ldots,n-1$ 
аралығындағы бүтін сандар болса, сөздікте кілт
ретінде кез келген дерек типін ала аламыз. Бұл ретте олардың жүйелелігі маңызды емес.

С++ стандартты дерекханасында жиынға ұқсас екі сөздік көрсетілімі бар. Олар:
\texttt{map} -- теңгерімделген бинарлы дараққа негізделген құрылым,
элементтерге қол жеткізу $O(\log n)$ уақыт алады, ал екіншісі -- \texttt{unordered\_map}, ол
хэшті қолданады және элементтерге қол жеткізу орташа есеппен $O(1)$ уақыт алады.
% the structure \texttt{map} is based on a balanced
% binary tree and accessing elements
% takes $O(\log n)$ time,
% while the structure
% \texttt{unordered\_map} uses hashing
% and accessing elements takes $O(1)$ time on average.

Келесі код кілті -- жолдар, ал мәндері -- бүтін сандар болатын
сөздік құрады:

\begin{lstlisting}
map<string,int> m;
m["monkey"] = 4;
m["banana"] = 3;
m["harpsichord"] = 9;
cout << m["banana"] << "\n"; // 3
\end{lstlisting}

Жолданған сауалда кілт болмаса, кілт автоматты түрде құрылып,
әдепкі қалпы бойынша мәнге ие болады. % default - по умолчанию - әдепкі қалпы бойынша
% If the value of a key is requested
% but the map does not contain it,
% the key is automatically added to the map with
% a default value.
Мысалы, келесі кодта ''aybabtu'' кілті 0 мәнімен қосылды.

\begin{lstlisting}
map<string,int> m;
cout << m["aybabtu"] << "\n"; // 0
\end{lstlisting}
\texttt{count} функциясы кілттің бар-жоғын тексереді:
\begin{lstlisting}
if (m.count("aybabtu")) {
    // key exists
}
\end{lstlisting}
Төмендегі код сөздіктегі барлық кілттерді мәндерімен бірге шығарады:
\begin{lstlisting}
for (auto x : m) {
    cout << x.first << " " << x.second << "\n";
}
\end{lstlisting}

\section{Итераторлар мен аралықтар}

\index{итератор}
С++ стандартты дерекханасындағы көптеген функциялар 
итераторлар көмегімен жұмыс жасайды.
Итератор -- деректер құрылымындағы элементке нұсқаушы.

Қолданыста ең жиі кездесетін \texttt{begin} 
мен \texttt{end} итераторлары деректер құрылымының барлық
элементтерін қамтитын аралыққа нұсқайды.
\texttt{begin} итераторы деректер құрылымының
ең алғашқы элементіне нұсқаса, 
\texttt{end} итераторы ең соңғы элементтен
\emph{кейінгі} позицияға нұсқайды.
Бұл ретте жағдай төмендегідей болмақ:
%The situation looks as follows:

\begin{center}
\begin{tabular}{llllllllll}
\{ & 3, & 4, & 6, & 8, & 12, & 13, & 14, & 17 & \} \\
& $\uparrow$ & & & & & & & & $\uparrow$ \\
& \multicolumn{3}{l}{\texttt{s.begin()}} & & & & & & \texttt{s.end()} \\
\end{tabular}
\end{center}

Итераторлардың асимметриясына назар аударыңыз:
\texttt{s.begin()} деректер құрылымының ішін меңзесе,
\texttt{s.end()} оның сыртына нұсқайды.
Сондықтан итераторлардың анықтаған аралығы -- \emph{жартылай ашық}.
% Thus, the range defined by the iterators is \emph{half-open}.

\subsubsection{Аралықтармен жұмыс}
Итераторлар деректер құрылымындағы элементтер аралығын қажет ететін
С++ стандартты дерекханасының функцияларында қолданылады.
% Iterators are used in C++ standard library functions
% that are given a range of elements in a data structure.
Әдетте біз деректер құрылымындағы барлық элементтерді 
өңдегіміз келеді. Сол себепті функцияда \texttt{begin} мен
\texttt{end} итераторлары беріледі.
% Usually, we want to process all elements in a
% data structure, so the iterators
% \texttt{begin} and \texttt{end} are given for the function.

Мысалы, келесі кодта вектор \texttt{sort} функциясы арқылы
сұрыпталады, кейін \texttt{reverse} функциясымен элементтер кері айналады, соңында \texttt{random\_shuffle} функциясының
көмегімен реттілікті араластырады.
% For example, the following code sorts a vector
% using the function \texttt{sort},
% then reverses the order of the elements using the function
% \texttt{reverse}, and finally shuffles the order of
% the elements using the function \texttt{random\_shuffle}.

\index{sort@\texttt{sort}}
\index{reverse@\texttt{reverse}}
\index{random\_shuffle@\texttt{random\_shuffle}}

\begin{lstlisting}
sort(v.begin(), v.end());
reverse(v.begin(), v.end());
random_shuffle(v.begin(), v.end());
\end{lstlisting}

Бұл функциялар қарапайым жиыммен де қолданыла алады.
Мына үлгідегі функцияларда итераторлар орнына жиымға нұсқаймыз:

\begin{lstlisting}
sort(a, a+n);
reverse(a, a+n);
random_shuffle(a, a+n);
\end{lstlisting}

\subsubsection{Жиын итераторлары}
Итераторлар жиын элементтеріне жету үшін жиі қолданылады.
Келесі код жиындағы ең кіші элементке нұсқайтын \texttt{it} итераторын
құрайды:
\begin{lstlisting}
set<int>::iterator it = s.begin();
\end{lstlisting}
Қысқартылған түрі:
\begin{lstlisting}
auto it = s.begin();
\end{lstlisting}
Итератор нұсқаған элементке \texttt{*} символын 
пайдалану арқылы қол жеткізе аламыз.
Мысалы, төмендегі код жиындағы алғашқы элементті шығарады:

\begin{lstlisting}
auto it = s.begin();
cout << *it << "\n";
\end{lstlisting}

Итераторды келесі операторлар арқылы жылжыта аламыз:
\texttt{++} (алдыға), \texttt{--} (артқа), яғни жиындағы дейінгі немесе келесі элементке жылжыту
деген мағынада қолданылады.

Келесі код элементтерді өсу ретімен шығарады:
\begin{lstlisting}
for (auto it = s.begin(); it != s.end(); it++) {
    cout << *it << "\n";
}
\end{lstlisting}
Ал мына код ең үлкен элементті шығарады:
\begin{lstlisting}
auto it = s.end(); it--;
cout << *it << "\n";
\end{lstlisting}

$\texttt{find}(x)$ функциясы мәні $x$ болатын элементке 
нұсқайтын итераторды қайтарады.
Дегенмен жиында $x$ болмаған жағдайда итератор 
\texttt{соңына} нұсқайды.

\begin{lstlisting}
auto it = s.find(x);
if (it == s.end()) {
    // x is not found
}
\end{lstlisting}

$\texttt{lower\_bound}(x)$ функциясы жиындағы мәні \emph{ең кемінде} 
$x$ болатын кіші элементке нұсқаушы итератор қайтарса, $\texttt{upper\_bound}(x)$ 
мәні $x$-тен \emph{үлкенірек} элементке нұсқайтын итератор қайтарады.
Екі функцияда да, егер сондай элемент болмаса, қайтарылған мән \texttt{end} болмақ.
Функциялар элементтер реттілігіне баса назар аудармайтын \texttt{unordered\_set}-те
қолданылмайды.

\begin{samepage}
Мысалы, келесі код $x$-ке ең жақын элементті анықтайды:

\begin{lstlisting}
auto it = s.lower_bound(x);
if (it == s.begin()) {
    cout << *it << "\n";
} else if (it == s.end()) {
    it--;
    cout << *it << "\n";
} else {
    int a = *it; it--;
    int b = *it;
    if (x-b < a-x) cout << b << "\n";
    else cout << a << "\n";
}
\end{lstlisting}

Код жиынның бос емес екенін жобалап, 
\texttt{it} итераторын қолданудың барлық мүмкіндіктерін тексеріп шығады.
Алғашында итератор ең кіші мәні кем дегенде 
$x$ болатын элементті көрсетеді.
Егер \texttt{it} \texttt{begin}-ге тең болса,
ол элемент жиындағы $x$-ке ең жақыны болмақ.
Ал егер \texttt{it} \texttt{end}-ке тең болса,
жиындағы ең үлкен сан $x$-ке ең жақыны болмақ.
Екеуіне де сәйкес келмесе, $x$-ке ең жақын элемент
не \texttt{it}, не оның алдыңғы жанындағы элемент болмақ.
\end{samepage}

\section{Басқа құрылымдар}

\subsubsection{Битсет}

\index{битсет}

Битсет (bitset) -- әр мәні 0 немесе 1 болатын жиым.
Мысалы, келесі код 10 элементтен тұратын битсет құрады:
\begin{lstlisting}
bitset<10> s;
s[1] = 1;
s[3] = 1;
s[4] = 1;
s[7] = 1;
cout << s[4] << "\n"; // 1
cout << s[5] << "\n"; // 0
\end{lstlisting}

Битсет қолданудың артықшылығы -- олардың қарапайым
жиынмен салыстырғанда жадыны азырақ талап етуінде. Себебі 
әр элемент жадыда 1 бит қана алады.
Мысалы, егер $n$ бит \texttt{int} жиымында сақталса,
жадыда $32n$ бит қолданылады, ал битсет тек $n$ бит
жадыны қажет етеді. Оған қоса, битсет мәндерін бит операцияларының арқасында 
тиімді қолдана аламыз. Бұл өз кезегінде биттік жиындарда орындалатын алгоритмдерді
оңтайландырады.

Төмендегі код битсет құраудың тағы бір жолын көрсетеді:
\begin{lstlisting}
bitset<10> s(string("0010011010")); // from right to left
cout << s[4] << "\n"; // 1
cout << s[5] << "\n"; // 0
\end{lstlisting}

\texttt{count} функциясы битсеттегі бірліктер санын қайтарады:

\begin{lstlisting}
bitset<10> s(string("0010011010"));
cout << s.count() << "\n"; // 4
\end{lstlisting}

Келесі кодта бит операцияларының қолданыс үлгілері көрсетілген:
\begin{lstlisting}
bitset<10> a(string("0010110110"));
bitset<10> b(string("1011011000"));
cout << (a&b) << "\n"; // 0010010000
cout << (a|b) << "\n"; // 1011111110
cout << (a^b) << "\n"; // 1001101110
\end{lstlisting}

\subsubsection{Екі жақты тізбек}

\index{екі жақты тізбек}
\key{Екі жақты тізбек} (deque) -- өлшемі екі жағынан да тиімді өзгеретін динамикалық жиым.
% is a dynamic array
% whose size can be efficiently
% changed at both ends of the array
Вектор сияқты екі жақты тізбекте де \texttt{push\_back} және \texttt{pop\_back}
функциялары орындалады. Сонымен қатар ол векторда жоқ \texttt{push\_front}
және \texttt{pop\_front} функцияларын да қамтиды.

Үлгіде екі жақты тізбек қолданысы берілген:
\begin{lstlisting}
deque<int> d;
d.push_back(5); // [5]
d.push_back(2); // [5,2]
d.push_front(3); // [3,5,2]
d.pop_back(); // [3,5]
d.pop_front(); // [5]
\end{lstlisting}

Ішкі құрылысы векторға қарағанда анағұрлым күрделірек, сол себепті
екі жақты тізбек вектордан баяуырақ болмақ.
Дегенмен екі ұшынан қосу және алу операциялары $O(1)$ уақыт алады.

\subsubsection{Стек}

\index{стек}

\key{Стек} (stack)
-- екі $O(1)$ уақыт алатын операцияларды, яғни элементті жоғарыға қосу және жоғарыдан алу 
операцияларын орындауға болатын құрылым.
Бұл жерде тек жоғарғы элемент қолжетімді болмақ.

Үлгідегі кодта стекті қолдану жолдары көрсетілген:
\begin{lstlisting}
stack<int> s;
s.push(3);
s.push(2);
s.push(5);
cout << s.top(); // 5
s.pop();
cout << s.top(); // 2
\end{lstlisting}
\subsubsection{Кезек}

\index{queue}

Кезек (queue) құрылымы да екі $O(1)$ уақыт алатын операцияларды, атап айтқанда 
элементті кезек соңына қосу  және кезектегі алғашқы 
элементті алу операцияларын қамтиды.
Мұнда кезектің соңғы және алғашқы элементтері ғана қолжетімді.

Үлгідегі кодта ол операцияларды қолдану жолдары көрсетілген:
\begin{lstlisting}
queue<int> q;
q.push(3);
q.push(2);
q.push(5);
cout << q.front(); // 3
q.pop();
cout << q.front(); // 2
\end{lstlisting}

\subsubsection{Басымдылық кезегі}

\index{басымдылық кезегі}
\index{үйінді}

Басымдылық кезегі (priority queue) элементтер жиынын жалғастырады және ол енгізу, кезек типіне байланысты
ең көп немесе ең аз элементті шығару және жою сияқты операциялардан тұрады.  
Енгізу мен жою $O(\log n)$ уақыт алса, шығару $O(1)$ уақыт алады.

Реттелген жиын (set) басымдылық кезегінің (priority queue) барлық операцияларын тиімді өңдегенімен,
басымдылық кезегін (priority queue) қолданудың айтарлықтай артықшылығы бар. Ол -- тұрақты факторлардың аздығы.
Басымдылық кезегі (priority queue) әдетте үйінді құрылымы арқылы орындалады және 
ол реттелген жиынды құрайтын теңгерімделген бинарлы дарақтан жеңілірек.

\begin{samepage}
Әдепкі қалпы бойынша, С++ басымдылық кезегінде элементтер кему ретінде сұрыпталады
және кезектен ең жоғары элементті тауып, оны жою мүмкіндігі бар.
Бұл келесі кодта келтірілген:

\begin{lstlisting}
priority_queue<int> q;
q.push(3);
q.push(5);
q.push(7);
q.push(2);
cout << q.top() << "\n"; // 7
q.pop();
cout << q.top() << "\n"; // 5
q.pop();
q.push(6);
cout << q.top() << "\n"; // 6
q.pop();
\end{lstlisting}
\end{samepage}

\newpage

Егер ең кіші элементтерді тауып, өшіретін 
басымдылық кезегін құрастырғымыз келсе,
оны төмендегідей ретпен орындай аламыз:

\begin{lstlisting}
priority_queue<int,vector<int>,greater<int>> q;
\end{lstlisting}

\subsubsection{Саясатқа негізделген деректер құрылымдары}

\texttt{g++} компиляторы 
C++ стандартты дерекханасында жоқ, кей өзге 
деректер құрылымдарын да қолдайды. Мұндай құрылымдар 
\emph{саясатқа негізделген} деректер құрылымдары деп аталады.
Оларды пайдалану үшін кодқа төмендегі жолдарды қосуымыз қажет.
\begin{lstlisting}
#include <ext/pb_ds/assoc_container.hpp>
using namespace __gnu_pbds; 
\end{lstlisting}
Осыдан кейін \texttt{set}-ке ұқсас, бірақ жиым сияқты индекстелетін
\texttt{indexed\_set} деректер құрылымы қолжетімді болады.
Мұндағы \texttt{int} мәндері келесі түрде анықталады:
\begin{lstlisting}
typedef tree<int,null_type,less<int>,rb_tree_tag,
             tree_order_statistics_node_update> indexed_set; 
\end{lstlisting}
Ал енді жиынды төмендегідей етіп құрай аламыз:
\begin{lstlisting}
indexed_set s;
s.insert(2);
s.insert(3);
s.insert(7);
s.insert(9);
\end{lstlisting}
Жиынның ерекшелігі -- сұрыпталған жиымдағы элементтерге индекстер арқылы қол жеткізу мүмкіндігінде.
% The speciality of this set is that we have access to
% the indices that the elements would have in a sorted array.
$\texttt{find\_by\_order}$ функциясы берілген позициядағы элементке итератор қайтарады:
% The function $\texttt{find\_by\_order}$ returns
% an iterator to the element at a given position:
\begin{lstlisting}
auto x = s.find_by_order(2);
cout << *x << "\n"; // 7
\end{lstlisting}
Ал $\texttt{order\_of\_key}$ функциясы берілген элементтің позициясын қайтарады:
% And the function $\texttt{order\_of\_key}$
% returns the position of a given element:
\begin{lstlisting}
cout << s.order_of_key(7) << "\n"; // 2
\end{lstlisting}
Егер элемент жиында болмаса, жиында болған жағдайдағы позициясын жобамен
аламыз:
% If the element does not appear in the set,
% we get the position that the element would have
% in the set:
\begin{lstlisting}
cout << s.order_of_key(6) << "\n"; // 2
cout << s.order_of_key(8) << "\n"; // 3
\end{lstlisting}
Екі функция да логарифмдік уақытта орындалады.

\section{Сұрыптаумен салыстыру}

Көп жағдайда есепті не деректер құрылымын, не 
сұрыптауды пайдалану арқылы шығара аламыз. Кейде бұл екі әдістің
тиімділіктерінде үлкен айырмашылықтар байқалады және  ол айырмашылықтар
уақытша күрделілігіне байланысты туындауы мүмкін.
% It is often possible to solve a problem
% using either data structures or sorting.
% Sometimes there are remarkable differences
% in the actual efficiency of these approaches,
% which may be hidden in their time complexities.

Мысал үшін $n$ элементтен тұратын $A$ және $B$ тізімдері берілген
есепті қарастырайық. Бізге екі тізімде де
кездесетін элементтер санын анықтау тапсырылады. Мысалы, 
\[A = [5,2,8,9] \hspace{10px} \textrm{және} \hspace{10px} B = [3,2,9,5],\]
тізімдері үшін жауап 3-ке тең. Себебі 2, 5 және 9 сандары екі тізімде де кездеседі.

Қарапайым шешімі: $O(n^2)$ уақытта барлық жұптарды тексеріп шығу.
Бірақ біз келесі кезекте тиімдірек болатын алгоритмдерді қарастырамыз.

\subsubsection{1-алгоритм}

$A$ тізіміндегі элементтерден тұратын жиын құраймыз, 
кейін $B$ элементтерімен өтіп шығып, әр элементтің $A$
тізімінде кездесуін тексереміз. Бұл тәсілдің тиімді болу себебі мынада:
$A$ элементтері жиында сақталған. \texttt{set} деректер құрылымын
қолдану нәтижесінде уақытша күрделілігі $O(n \log n)$ болмақ.

\subsubsection{2-алгоритм}

Реттелген жиын қолдану міндетті емес болғандықтан
\texttt{set} құрылымын \texttt{unordered\_set}
құрылымымен алмастыра аламыз. Бұл -- алгоритмді тиімдірек етудің 
жеңіл жолы. Себебі бізге негізгі деректер құрылымын өзгерту
ғана жеткілікті. Жаңа алгоритмнің уақытша күрделілігі -- $O(n)$.
% This is an easy way to make the algorithm
% more efficient, because we only have to change
% the underlying data structure.
% The time complexity of the new algorithm is $O(n)$.

\subsubsection{3-алгоритм}

Деректер құрылымдарының орнына сұрыптауды қолдана аламыз. 
Біріншіден, $A$ және $B$ тізімдерін сұрыптаймыз. Кейін, 
екі тізімді де бір уақытта өтіп шығу арқылы бірдей элементтерді анықтаймыз.
Сұрыптаудың уақытша күрделілігі $O(n \log n)$-ге тең және алгоритмнің қалған бөлігі
$O(n)$ уақыт алатындықтан қорытынды уақытша күрделілігі $O(n \log n)$ болмақ.

\subsubsection{Тиімділікті салыстыру}

Келесі кестеде жоғарыдағы алгоритмдердің 
қаншалықты тиімді екендігі $n$ өзгеріп отыратын және 
тізім элементтері $1 \ldots 10^9$ аралығындағы
кез келген бүтін сандар болатын жағдайлар арқылы көрсетілген:
% The following table shows how efficient
% the above algorithms are when $n$ varies and
% the elements of the lists are random
% integers between $1 \ldots 10^9$:

\begin{center}
\begin{tabular}{rrrr}
$n$ & 1-алгоритм & 2-алгоритм & 3-алгоритм \\
\hline
$10^6$ & $1.5$ s & $0.3$ s & $0.2$ s \\
$2 \cdot 10^6$ & $3.7$ s & $0.8$ s & $0.3$ s \\
$3 \cdot 10^6$ & $5.7$ s & $1.3$ s & $0.5$ s \\
$4 \cdot 10^6$ & $7.7$ s & $1.7$ s & $0.7$ s \\
$5 \cdot 10^6$ & $10.0$ s & $2.3$ s & $0.9$ s \\
\end{tabular}
\end{center}

1 және 2-алгоритмдердің айырмашылығы жиын құрылымдарында ғана байқалады. 
Бұл есепте жасалған таңдаудың өңдеу уақыты үшін әсері орасан зор. 
Себебі 2-алгоритм 1-алгоритмге қарағанда 4-5 есе жылдамырақ.
% Algorithms 1 and 2 are equal except that
% they use different set structures.
% In this problem, this choice has an important effect on
% the running time, because Algorithm 2
% is 4–5 times faster than Algorithm 1.

Дегенмен ең тиімді алгоритм -- сұрыптау қолданатын 3-алгоритм.
Ол 2-алгоритммен салыстырғанда 2 есе аз уақыт жұмсайды.
% It only uses half the time compared to Algorithm 2.
Қызығы, 1-алгоритм мен 3-алгоритмнің екеуінің де уақытша күрделілігі -- $O(n \log n)$,
бірақ бұған қарамастан, 3-алгоритм он есе жылдамырақ.
Мұны келесі фактілер арқылы түсіндіре аламыз: 
сұрыптау -- қарапайым процедура және ол 3-алгоритмнің
басында  1 мәрте орындалады, ал алгоритмнің қалған бөлігі
сызықты уақытта жұмыс жасайды. Басқа қырынан қарасақ, 
1-алгоритм жұмыс барысында күрделі теңгерімделген бинарлы дарақты сақтайды.
% Interestingly, the time complexity of both
% Algorithm 1 and Algorithm 3 is $O(n \log n)$,
% but despite this, Algorithm 3 is ten times faster.
% This can be explained by the fact that
% sorting is a simple procedure and it is done
% only once at the beginning of Algorithm 3,
% and the rest of the algorithm works in linear time.
% On the other hand,
% Algorithm 1 maintains a complex balanced binary tree
% during the whole algorithm.
