\chapter*{Алғысөз}
\markboth{\MakeUppercase{Алғысөз}}{}
\addcontentsline{toc}{chapter}{Алғысөз}

Кітаптың мақсаты -- өз оқырмандарын спорттық бағдарламалаудың тұжырымдарымен толық таныстырып шығу. Сіз бағдарламалаудың негіздерін  бұған дейін де біледі деп бағамдаймыз, десек те, спорттық бағдарламалауда қандай да бір алдын ала даярлық қажет емес деп болжанады.

Кітап, әсіресе, алгоритмдерді үйренгісі келетін және Информатика бойынша халықаралық олимпиадаға (IOI) немесе  Студенттердің бағдарламалау бойынша халықаралық конкурсына (ICPC) қатысқысы келетін білім алушыларға арналады. Сондай-ақ спорттық бағдарламалауға қызығушылық танытатын жалпы жұртшылыққа да пайдалы болмақ.

% The purpose of this book is to give you
% a thorough introduction to competitive programming.
% It is assumed that you already
% know the basics of programming, but no previous
% background in competitive programming is needed.

% The book is especially intended for
% students who want to learn algorithms and
% possibly participate in
% the International Olympiad in Informatics (IOI) or
% in the International Collegiate Programming Contest (ICPC).
% Of course, the book is also suitable for 
% anybody else interested in competitive programming.

Бәсекеге қабілетті, жақсы спорттық бағдарламалаушы болу үшін көп уақыт қажет. Алайда мұны біраз дүниелерді үйрену үшін берілген мүмкіндік деп қабылдаған жөн. Егер уақытыңызды кітапты оқуға, есептерді шығаруға және конкурстарға қатысуға бөліп жүрсеңіз, алгоритмдер туралы жақсы жалпы түсінік алатыныңызға сенімді бола беріңіз.

% It takes a long time to become a good competitive
% programmer, but it is also an opportunity to learn a lot.
% You can be sure that you will get
% a good general understanding of algorithms
% if you spend time reading the book,
% solving problems and taking part in contests.

Кітап үнемі жетілдіріліп, зерделеніп отырады. Кітап туралы пікіріңізді сіз әрқашан \texttt{ahslaaks@cs.helsinki.fi} мекенжайына жібере аласыз (ред. ескерту: қазақ тіліндегі аудармасы бойынша ұсыныстарыңызды, кітапшаның қазақша мазмұнына қатысты ойларыңызды \texttt{cphb.kz@gmail.com} мекенжайына жолдай аласыз).

% The book is under continuous development.
% You can always send feedback on the book to
% \texttt{ahslaaks@cs.helsinki.fi}.

\begin{flushright}
Хельсинки, 2019 ж. тамыз \\
Лааксонен Антти
\end{flushright}


% \begin{flushright}
% Helsinki, August 2019 \\
% Antti Laaksonen
% \end{flushright}

